\section{Sequencing data}
\subsection{Next-generation and third generation sequencing}
\subsection{Reads}

\section{\kmers and how to store them}

\section{Pangenomics and pangenome graphs}
\label{sec:background_pangenomics}
Since the beginning of genomics, all analysis based on sequencing data depended upon the use of a single linear reference genome, i.e. the best assembled genome available for a species, to extract useful information from the DNA. We now know that this approach is suboptimal in a wide range of applications as a lot of genetic material of the species cannot be present in a single linear refererence: this is valid for eukariotes and even more for bacteria.  
This limitation at the beginning was not solvable due to the scarcity of high quality assembled genomes as the technologies of sequencing and computational tools were not mature enough. For example, the Human Genome Project took 13 years to produce its result~\cite{humangenomeproject} and the absence of long reads with decent error rate made it impossible to automatically resolve repetitive regions like telomeres and centromeres~\cite{human-pangenomics-era}, producing a reference only $92\%$ complete~\cite{t2t}. This problem was only solved in 2022~\cite{t2t}. \\
Right now we are witnessing a real revolution in the sequencing. As the price is significantly lowering, also thanks to competition of new companies entering the market, new scientific discoveries and technological advances are leading to a remarkable increase of quality, in term of per-base error rate, and throughput. This means than right now we dispose of a rich wealth of high quality sequencing information to produce hundreds or thousands of new first grade assemblies.
This progress lead to a shift in paradigm with increasing effort from the scientific community to propose new methods to analyse one or multiple genomes: not anymore by comparing it against a single reference sequence but against a comprehensive representation of the species. \\
This novel way to overcome the limits of "linear genomic" and consider all the variation in a single species is called pangenomics. \\
Various efforts are being made on producing reference pangenomes of yeasts, bacterias, plants and animals, including humans. In order to do so, new tools to construct and then analyse and use such representations are being developed. 
It is important here to notice, as it will be stressed in the next sections and chapters, that construction is just the first step and that is very important to understand and work on which are the operations that can be succesfully performed by these representations. \\

\section{Graphs}

\subsection{De Bruijn Graphs}
\subsubsection{Colored and Compacted De Bruijn Graphs}
\subsection{Variation Graphs}

