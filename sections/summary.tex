 \chapter{Summary}
\label{sec:Summary}
\section{English summary}
For over two decades, the human reference genome has laid the ground for human genomic research. However, its power to provide insights has been constrained by the presence of gaps and simulated sequences. In 2022, the Telomere-to-Telomere consortium achieved an important milestone by releasing the first full sequence of an haploid human genome (T2T-CHM13), empowering a new discoveries on the previously missing regions of the genome. Nevertheless, a single genome cannot adequately represent the entire genetic diversity within the human population, in particular large \emph{structural} variants.\\
To address the inherent reference bias of using a single genome as mean of comparison, the scientific community is transitioning towards pangenomes: these are models that encapsulate multiple alleles from a collection of genomes. The field of computational pangenomics aims at finding new and more efficient pangenome models that can improve the results of reference-based analyses. Among others, the most common pangenome representation is based on graphs. \\
This dissertation presents two primary contributions to computational pangenomics. The first is a comparative analysis of pangenome graph representations, based on the construction of the largest pangenome graph to that date. This analysis compares different graph models, using five state-of-the-art tools, shedding light on key differences between the representation, particularly on how they capture genetic variation in complex loci.\\
The second contribution focuses on advanced data structures for \kmer sets representation. In particular, on three novel data structures that focus on improving metadata association, downstream analysis accessibility and scalability. These \kmer-based methods aim at facilitating genomic and pangenomic analyses.\\
This dissertation present my contributes to the ongoing evolution of computational pangenomics research.
%The human reference genome, which has served as the foundation of genomics for over 20 years, contains gaps and computationally simulated sequences, limiting its usefulness. The Telomere-to-Telomere consortium recently completed the first full sequence of a haploid human genome (T2T-CHM13), improving understanding of genomic variations. However, a single genome cannot represent the full genetic diversity of the human population, as structural variants are often missed. To address reference bias, a shift toward pangenomes, which are models that capture multiple alleles from a collection of genomes, is being pursued. These pangenomes can be represented using graph-based models, and this work contributes to computational pangenomics by analyzing different tools for building such graphs and by proposing new data structures for efficiently handling \kmer sets, which offer advantages for scalability, speed, and flexibility in genomic analysis.
%This dissertation focuses on computational pangenomics and proposes two main contributions. The first is an analysis of the largest human pangenome constructed at the time, comparing pangenome graph representations using five advanced tools. The analysis highlights differences in how these tools capture genetic variations, assisting users in selecting the best graph type for their needs. The second contribution involves improving data structures for representing \kmer sets and their representation of metadata. \kmer-based data structures offer advantages for genomic analysis in terms of scalability, speed, and memory efficiency, and the dissertation focuses on my contribution on three new data structures that advance the current state of \kmer representations.
\newpage
\section{Résumé en français}
Depuis plus de vingt ans, le génome Humain de référence a jeté les bases de la recherche en génomique Humaine. Toutefois, la quantité d'informations extraite de cette référence est limitée par sa non complétude et ses morceaux issus de simulations. En 2022, le consortium Telomere-to-Telomere a franchi une étape importante en publiant la première séquence complète d'un haploïde Humain (T2T-CHM13), ce qui a permis de faire de nouvelles découvertes sur les régions du génome qui étaient auparavant manquantes. Néanmoins, un seul génome ne peut pas représenter de manière adéquate l'ensemble de la diversité génétique au sein de la population Humaine, en particulier les grands variants \emph{structuraux}.\\
Pour remédier au biais de référence inhérent à l'utilisation d'un seul génome comme moyen de comparaison, la communauté scientifique s'oriente vers les pangénomes : il s'agit de modèles qui englobent de multiples allèles provenant d'une collection de génomes. Le domaine de la pangénomique computationnelle vise à trouver de nouveaux modèles de pangénomes plus efficaces qui peuvent améliorer les résultats des analyses basées sur les références. Entre autres, la représentation la plus courante du pangénome est basée sur les graphes.\\
Cette thèse présente deux contributions principales à la pangénomique computationnelle. La première est une analyse comparative des représentations graphiques du pangénome, basée sur la construction du plus grand graphique du pangénome à ce jour. Cette analyse compare différents modèles de graphes, en utilisant cinq outils de pointe, mettant en lumière les principales différences entre les représentations, en particulier sur la façon dont elles capturent la variation génétique dans les loci complexes.\\
La deuxième contribution porte sur les structures de données avancées pour la représentation des ensembles de \kmer. En particulier, trois nouvelles structures de données visent à améliorer l'association des métadonnées, l'accessibilité des analyses en aval et la scalabilité. Ces méthodes basées sur les kmer visent à faciliter les analyses génomiques et pangénomiques.\\
Cette thèse présente ma contribution à l'évolution en cours de la recherche en pangénomique computationnelle.