\chapter{Preface}
This thesis is submitted in partial fulfillment of the requirements for the degree of \emph{Philosophiae Doctor} at \sorbonne.
The research presented here was conducted at the \pasteur, under the supervision of Dr. Rayan Chikhi and Dr. Yoann Dufresne.
This work is part of the Marie Skłodowska-Curie ITN ALPACA project that has received funding from the European Union’s Horizon 2020 research and innovation program under the Marie Skłodowska-Curie grants agreements No \num{956229} and \num{872539}.

This thesis is a collection of some of the different projects I worked on during my stay at \pasteur. I begin with an small foreword of the research output of my PhD, and a gentle introduction of the scientific background needed to contextualize the work proposed in the manuscript. In the first part I present the published paper I am first author of, together with other unpublished work I lead or independently developed. In the second part I present other results of my scientific production, with novel elaboration of the work that appeared in the other papers I am co-author and presentation of projects that have or will be submitted to revision.
The common theme is human pangenomics and computational methods used to generate and use such models to infer relevant information.  
This essay ends with a chapter showcasing future perspectives and conclusions.

\section*{Acknowledgements}
First and foremost, I would like to express my deepest gratitude to Rayan for his constant support, patience, encouragement, and understanding throughout this journey. His ability to bring value to our work, his admirable work ethic, and the scientific intelligence he demonstrated in all the fields we discussed have been a true source of inspiration. I am truly fortunate to have had the opportunity to learn from him.\\
I would also like to extend my deepest gratitude to Yoann, who has been a constant source of motivation and inspiration throughout this journey. His passion for his work has been truly contagious, fueling my own interest in this field. Yoann not only helped me significantly improve my computer science skills but also became a valued person in my life in Paris. Whether it was inviting me to events at La Treso, patiently teaching me French alongside Pierre, explaining me the technicalities of the game he invents, talking about politics or inviting me for runs, his presence enriched both my personal and professional experiences.\\
A heartfelt thanks goes to François Sabot and Matthias Zytnicki for kindly agreeing to review this manuscript and for devoting their time and effort to being part of the thesis jury. I understand the significant commitment involved, and I deeply appreciate their contributions.\\
I extend my sincere thanks to Paola Bonizzoni, Camille Marchet, and Pierre Peterlongo for serving as members of the jury. Each one of them has played a special role in the trajectory of my thesis for particular reasons.\\
Additionally, I wish to express my appreciation to Pierre Peterlongo and Martin Weigt for being part of my Comité de Thèse, offering valuable feedback that greatly contributed to the successful completion of this project.\\
To the SeqBio group, I would like to extend a special thank you. The group is composed of remarkable individuals, and I am lucky to have been part of it. In particular, I want to thank Camila for always being there to talk things through and for offering invaluable advice. I am equally grateful to Yoshihiro for his kindness and willingness to help in any circumstance and to Mélanie, whose support during every “chat noir” situation ensured I could finish this PhD.\\
I want to express deep gratitude to the Italian and Spanish community in Pasteur, who made me feel at home in every moment of these 40 months.\\
Lastly, I would like to thank the Alpacas, a distinguished group of young scientists, but even more importantly, incredible people. It has been such a pleasure to work and spend time with all of you.\\

%Finalmente, quiero agradecer a mi Belsy,\\
%Voglio inoltre ringraziare la mia famiglia. \\
%A mia madre, che mi ha sempre insegnato a rispettare gli altri e ad essere una persona corretta in questo mondo. Che ha speso anima e corpo quando non sono stato bene e che si preoccupa ancora, come ogni madre italiana che si rispetti, di suo figlio.\\
%A mia sorella, che mi ha insegnato a non accontentarmi e a puntare sempre in alto.\\
%A mio padre, che mi ha trasmesso la passione per l'informatica.\\
%A mia nonna, il cui amore e bonta' mi riempiono il cuore ogni volta in cui stiamo assieme. \\
%A mio nonno, che se ne e' andato poco prima che partissi per questo dottorato, e che rimarra' sempre nel mio cuore.\\
%I would finally like to thank my friends, in Italy, Paris and abroad, who took care of me and made me enjoy my life. \\
\vskip\onelineskip
\begin{flushleft}
    \sffamily
    \textbf{\theauthor}
    \\
    Paris,\MONTH\the\year
\end{flushleft}