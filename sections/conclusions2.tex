\chapter{Conclusions}
%\label{sec:conclusions}
%This manuscript is the result of 40 months of doctoral journey. The common theme is the use of \kmer based algorithms and tools to tackle complex genomic problems that consider multiple different samples at the same time. This effort was always driven by the desire to produce software or analyses that would help answer biological questions.\\
%It hasn’t been a straightforward path as many projects I embarked in did not get to a stage of meaningful contribution to the scientific community. Nevertheless such efforts taught me important lessons about working alone or with other colleagues and strengthened my understanding of other genomics and bioinformatics fields.
%I reckon that the scope of this doctoral thesis might seem quite broad, starting with an analysis of computational pangenomics methods for human genomes representations and ending with data structures to represent \kmers in a cache efficient way. In my view this is the result of constant curiosity about the whole sequence bioinformatics problems tackled by the team where I was and the reflection of a comprehensive and multifaceted approach to the challenges of the field.\\
%We proposed an analysis on the construction and representation of pangenome graph from high quality human haploid assemblies was well received by the community as it shed light on the characteristics of both their internal representation and the methods to generate them. It stresses the importance of selecting the kind of graph that best fits the particular application, specifically in the way it represents variations in the DNA sequence of the individuals. Variation graphs are better to perform specific downstream analysis and are more intuitive to understand, manipulate, visualize and analyze while de Bruijn Graphs are more efficient to generate, scale better and give guarantees on the preservation of the input sequences. Finally, this work stresses the difficulties of proposing a one-fits-all solution and points out areas of research that would make \kmer based approaches more attractive to the genomic community. I believe pangenomics is the key of solving many issues and deficiencies present in current genomics approaches: it is a novel area that is evolving now and will need much more effort to produce viable solutions to all the genomics tasks biased by the use of a single reference sequence.\\
%My work on the Backpack Quotient Filter has been focused on coding the underlying data structure, the Rank Select Quotient filter, in a way that could be easily used and manipulated for different high level interfaces that would implement different filters. The final method proposed on top of this data structure a particular encoding of the counting information together with the integration of the Fimpera scheme for \kmer storing and retrieving. My work lied on the demonstration of the efficacy of the implementation I wrote by recreating on top of it the Counting Quotient Filter as originally conceived together with the Fimpera scheme. This demonstrated that implementation differences affect the magnitude of the data such data structures can analyze, even if the information provided as output is the same. This work served as confirmation that  improvements in the representation ok \kmer sets can drastically change the way in which large data collections can be interrogated and explored, even if very complex. In order to improve in the future such analysis power, new tools will require enhanced methods and refined coding techniques to exploit the maximum out of computational power to analyze exponentially increasing amounts of data.\\
%Finally, I gave my contribution to other projects, like \muset, that is a pipeline for the construction of uniting matrices (both abundance and presence/absence). This is the first effort to produce such data structure and is an advancement from \kmer matrices, that contained almost the same information while being more complex to analyze. This data structure can serve different applications, from pangenomics to metagenomics data as well as cancer transcriptomics, as it provides a very clear representation of the genetic content of multiple samples and their difference that is easily accessible by data analysis and machine learning tools.\\
%The tools, pipelines and data structures I analyzed or coded are still not the definitive answer to any of the problems that pangenomic and metagenomics still face but are a clear demonstration of what are good steps in the right direction. In the future there is plenty of work to do. In the pangenomic field, methods to demonstrate the biological correctness of pangenome graphs, like it is now done with assemblies, as well to scale pangenomic representation to handle large collections of eukaryotes.
%In metagenomics new methods will need to follow the pace of the rapid increase of data that is being sequence everywhere in the world.\\
%Finally, all the methods that will need to analyze and compare multiple individuals or samples will need to use \kmers as part or as bedrock of their development, as I think this thesis proved, are a simple, versatile and effective way of representing DNA sequences.

%As the understanding of public metagenomic repositories is important when developing software that is meant to encode their informations, I also helped in the writing of a section about such repositories in a method primer review. My contribution to this large collaborative effort was to provide to potential users a first, concise and expert overview on which are the main efforts to store, categorise and analyse public metagenomics data.\\ 
